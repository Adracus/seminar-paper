\chapter{Summary}

Summary ...

\chapter{Future work}

\section{Primes with Miller Rabin}

In the current implementation of Duse, Mersenne primes are used to
compute in finite fields. They are easy to compute and there are
plenty of numbers lower than the used ones. An issue which results
of the Mersenne Primes is the predictability of the prime number
which is used for encryption. Also, encryption is limited to secrets
which are smaller than the largest prime number. If one wants to
get a larger prime number than the largest mersenne prime number the
computation time rises drastically. This is because all numbers
$< \sqrt{n}$ (where $n$ is an uneven number being tested) have to be evaluated
as possible dividers of $n$. If $n$ is not a prime, $n+1$ has to be
tested until a prime number is found. The computation time can be
decreased by applying a special version of the Sieve of Erathostenes but
it is still way too high. This is where the \textit{Miller-Rabin Test}
comes in. The Miller-Rabin Test is a Monte-Carlo algorithm. If one inputs
a number $n$, the tests has two possible results: It either returns
that $n$ is not a prime number or that $n$ is propably a prime number.
Internally, the test consist of congruency characteristics of prime numbers.
The test can be repeated and the propability of a number not being a
prime being accepted as prime is $\frac{1}{4^s}$, where s is the number
of iterations of the Miller-Rabin Test.

Although it is not as secure as exact computation, the Miller-Rabin Test
is also used in the RSA internals to get high prime numbers for the
private- and public key generation. RSA is considered as a secure
encryption scheme (also with secure paddings etc.) and if such a tested
software can rely on prime numbers generated by using the miller rabin
algorithm, Duse can too.
\section{Private key sharing}
\section{Updating key pairs}
\section{Offline access}
\section{Advanced permissions}
\section{RSA OAEP}
\section{Key establishment}
\subsection{Diffie-Hellman key exchange}
\subsection{Public key infrastructure}
\subsection{Web of trust}
\section{Mark as insecure (either key of user compromised or secret itself compromised)}
\section{Expiry date}
\section{Client side signature checks}
\section{Using symmetric encryption in conjunction with the current system}
